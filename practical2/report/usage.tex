To build our solver it is sufficient to run the included Makefile using the
\texttt{make} command. This assumes that Sylvan and all its dependencies are 
correctly installed and on the \textsc{path}. In particular it 
assumes that the file \texttt{libsylvan.a} is on the \textsc{path}, it will also
search for \texttt{libsylvan.a} in \texttt{./sylvan/src/} if it is not on the
path.

After building our solver it is possible to run the solver using 
\texttt{./main <Sokoban sreen>}. It is possible to either give a path to a
Sokoban screen or to input a screen directly. Example~\ref{exe:screen2000} will
run our solver for the file \texttt{/screens/screen.2000}.
\begin{equation}\label{exe:screen2000}
	\mathtt{\$}\text{ }\mathtt{./main}\text{ }\mathtt{/screens/screen.2000}
\end{equation}

When running the implementation it will output some diagnostic information to
\texttt{std::cerr} and will state that it is not completely implemented on 
\texttt{std::cout}. 

Some properties in our implementation are not yet based on the inputed screen, 
but our instead based on a minimal version, one with all the outside walls 
removed and an extra wall in the top left corner, of screen 2000. This file is 
also included as \texttt{screen.2000\_minimal}. 