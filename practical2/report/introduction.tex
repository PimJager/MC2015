Reachability analysis is the process of computing the set of reachable states 
of a system. This can be used to build a model checker for a system, which can
determine whether or not a system satisfies some given given properties. This
is particularly useful for systems which have a lot of possible states.

In this assignment we will use reachability analysis, or more precise, the 
symbolic reachability algorithm presented in the lectures, to determine whether
or not it is possible to solve a certain Sokoban screen.

The symbolic reachability algorithm requires that states can be expressed as
logical formulas. To do this in an efficient way we will use \robdds, which 
are an efficient representation for logical formulas with a large number of
variables, both for storage and for operations.

In our implementation we use the \robdd library Sylvan. Our implementation
is written in C++.