\subsection{Encoding}
In our implementation Sokoban screens are encoded using variables denoting the
x and y position of each block and the man. Initial, error and goal properties
are then defined in terms of these variables, as is the transition relation. 
Below we will show this more 
clearly and more formally and we will also link the formal specification to the
implementation in Sylvan.

Ultimately we will combine these parts to check that there is a path from the
initial configuration to a configuration which satisfied the goal property 
without ever satisfying the error property. More formally, we will check the
following formula:
\begin{equation}\label{eq:eg_end}
	\mathsf{EG (\neg error \cup goal)}
\end{equation}

//Nog iets over hoe we screen leyouten met index voor columns en rows enzo

\paragraph{Positions} For the man and for each block variables are defined which
indicate in which row and in which column the block/man is. For a screen with
$x$ columns, $y$ rows and $n$ blocks we define $(n+1)*(x+y)$ 
variables. 

For each block we define
\begin{itemize}
	\item[$bx_{i,x}$] which indicates if block $i$ is in the column numbered $x$.
	\item[$by_{i,y}$] which indicates if block $i$ in in the row numbered $y$.
\end{itemize}
For the man we define similar variables, but named just $mx_x$ for the columns
and $my_y$ for the rows as a screen only has one man.

The actual implementation defines just a sequence of \textsc{bdd}vars, one for 
each of these variables. 

\paragraph{Transition relation}
Before showing the different properties of the system, we will first show the 
transition relation which describes how the man moves around the board and how
this affects blocks around him.

The main idea behind the transition relation is rather simple. Given that the 
man is in a certain position $(x,y)$ he can move either up, down, left or right,
and if there is a block in that position that block moves in the same direction.
If the man is currently at an edge then he can not move in the direction which 
would make him fall of the board, and if the man is separated from the edge of 
the board by only a block, then he also can not move in that direction as that
would push the block outside of the board.

Any transitions which result in either the man or a block overlapping with a 
wall, or which result in two blocks overlapping are considered erroneous and
their resulting states will satisfy the error property.

Equation~\ref{eq:transition} shows the formal transition relation. 
$\mathsf{blocks}$ is considered the set of block(numbers) in this screen, $rows$ 
the set of row numbers and $cols$ the set of column numbers. $max_r$ will be
the maximum column number and $max_c$ the maximum column number. In this 
relation the primed version of each variable denotes that variable in the next
state, i.e. the state after the transition has happened.
We will only 
fully show the formula for moving right, the formulas for moving left, up and
down are defined similar. 

\label{eq:transition}
\begin{multline}
	\bigwedge_{x \in cols, y \in rows} (mx_x \wedge my_y) \implies \bigg(\\  
		\Big( x \neq max_c \wedge \big(x \neq (max_c-1) 
			(\wedge_{b \in blocks} \neg (bx_{i,x+1}) \wedge by_{b,y})\big)\\  
		\implies \big( \neg mx_{x}' \wedge mx_{x+1}' \wedge my_y' \\  
			(\wedge_{b \in blocks} bx_{b,x+1} \wedge by_{b,y} \implies
				bx_{b,x+1}' \wedge bx_{b,x+2}' \wedge by_{b,y}' )\big) \Big)\\  
		\vee \Big(\text{similar for left}\Big) 
			\vee \Big(\text{similar for up}\Big)
			\vee \Big(\text{similar for down}\Big) \bigg)
\end{multline}
For standard screens this encoding is actually more complex than necessary. 
Since all screens are bordered by a layer of walls it is not necessary to check
if a block would be pushed of screen and if the man would walk of screen, since
states in which a transition could result in this would already satisfy the 
error property since a block or the man would overlap with the outside walls. 
However this transition allows for an optimization in which all layers of 
outside wall have been removed from a board, reducing the state space.